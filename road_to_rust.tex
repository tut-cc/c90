\chapter{Rustへの道}
\begin{quotation}
  Program testing can be a very effective way to show the presence of bugs, \\
  but is hopelessly inadequate for showing their absence. ...

  The programmer should let correctness proof and program grow hand in hand.
\end{quotation}
\begin{flushright}
  --- Edsger W. Dijkstra,``The Humble Programmer'' (1972)
\end{flushright}

\section{前書き}
この種の同人誌に於ける利点は,何と言っても全く意識外の情報を仕入れる機会が得られる所だろう.
すると弊サークルのプログラミング言語大好きおじさんこと, @\_Nnwww が貴方に寄せる本記事の内容は1つしか無い,急進にして注目のプログラミング言語紹介である.
まぁ急進にして注目のといえば人の属する界隈によりけりだが,ここでは正式リリースから1年が経過し,
2016年にstack Overflowが行ったデベロッパー向けの調査にて「最も愛されている言語」に登りつめたプログラミング言語\cite{most_loved},
「Rust」について紹介しよう.
即ち,本記事はRustが現れた文脈を共有しない読者に向け,その理念や設計を示すものである.

なお,貴方がRustに今すぐ入門する意気に溢れているならばこの記事を読むよりも
rustup\cite{rustup}を導入し,
有志による翻訳リポジトリ\cite{rustdoc_ja}(拙訳も提供させていただいた)及びRustnomicon\cite{nomicon}の読破を推奨する.

間違い等については前述のアカウントに御一報下さい.文体ですか?やってみたかっただけです :)

\section{Rustとは}
Rustは「ゼロコスト抽象化」を実現する「トレイト」ベースの静的型付き「手続き型プログラミング」言語である.完!

というのは言葉足らずの私の主観だが事実しか並んでいない筈だ.
もう少し真面目に話せば1400人以上ものコントリビュータによって開発されているLLVMバックエンドのネイティブコンパイル言語であり,
Mozillaが製作中の新作webレンダリングエンジンServoの実装に用いられている.
公式では安全性,速度.並行性の3つの目標にフォーカスし,それらをガーベジコレクタ無しで実現するシステムプログラミング言語である,といった紹介がなされており,
レイヤ的にはこれまでC/C++のような言語が取り組んでいた領域に昨今の知見を持ち込む進捗だと言えるだろう.

さて,少なくともこの言語の場合,来たる知見の方角を意識しつつ機能を咀嚼するとわかりやすい.といっても大別して2つだと考えられる.
即ち「関数型プログラミングから来た知見」と「同レイヤの先達 (主にC++)にて醸成された知見」である.それでは見ていこう.

\section{ゼロコスト抽象化}
実行時のコストを支払わずに強力な抽象化や安全性検査等を実現するという思想.
力の限り不適切に形容すると「あ〜HaskellがC言語並の実行速度で動かないかなぁ」という考えであり,プログラミングの理想主義である.
特定の機能というより普遍的に現れるRustの指針であり,言語機能から標準ライブラリの設計,果てはコンパイルの最適化まで及ぶ.
従ってコードを書く際のつらみも,書かれたコードのつよみも,大凡ここから発露するものである.

\subsection{トレイト}
C++やJava,C\# に代表されるジェネリクスについて思い出して欲しい.
型をパラメータTに置き換える事により,逐一具体的な型についてコピーアンドペーストしなくとも
汎ゆる型を扱う事の出来る凄い奴だ.
だが,これらジェネリクスの型パラメータはそれがどういった性質の型であるか制約を課す事が出来ない.
そのため「ジェネリックな関数内のどの式でエラーが起きたか」は知ることが出来るが,
「その型にどういった性質が欠如していたのか」を具体的に知ることは出来ないのである.
\footnote{C++ではテンプレートメタプログラミングに依るConceptという手法で実現可能.
  \\ 言語機能としての実装が熱望されているが,C++11の頃から今日に至るまで延期され続けている...}

するとジェネリクスについて我々は考えを改めなければならない.
我々が必要なのは「汎ゆる型」ではなく,「特定の性質を持つ汎ゆる型」なのである.

トレイトとはこの語の和訳の通り,ある型が有する「特質」乃至は「特徴」を表す言語機能である.
此処で言う特質とは特定のシグネチャの関数群や関連する型が適切に実装されている事を指す.

簡単な実例を見てみよう.(あくまで例であり,標準ライブラリではもう少し複
雑である.)
\begin{lstlisting}[language={C++},caption=Hash可能である事を示すトレイト,label=hash_t]
trait Hashable where Self : Eq {
    fn hash(&self) -> u64;
}
\end{lstlisting}
これはとある型がハッシュ可能である事を示すトレイトだ.ここで対象となる型をSelf,
その値をselfと表現する.
Rustでは大文字から始まるのが型やトレイトで,
小文字で始まるのがそれ以外の値やシンボルだと捉えて貰って構わない.

このHashableトレイトはSelf型の値の参照を取り,符号無し64ビット整数型を
返す関数hashを要求する.また,トレイト名に続く where句 という部分で,
対象となる型はEqトレイトが実装されていなければならないという制約を追加している.
これに対し簡単な実装を与えてみよう.

\begin{lstlisting}[language={C++},caption=トレイトの実装と利用例,label=hash_use]
impl Hashable for bool {
    fn hash(&self) -> u64 {
        if *self { 0 } else { 1 }
    }
}

impl Hashable for i64 {
    fn hash(&self) -> u64 {
        *self as u64
    }
}

struct HashTable<K: Hashable, V> { ... }
\end{lstlisting}

impl 実装するトレイト for 対象の型 という形で要求された関数や追加の型等を実装す
る.ここではbool型を0と1へ,符号付き64ビット整数型をビット表現で等価な符号無しへ
変換している.すると利用例として最後の行のような構造体が定義できる(実装は省く).
ハッシュテーブルのキーと取り扱う値をそれぞれK, Vとして多相化し,
Kについてはハッシュとしての特性を持つ型に制約を与えている.
より複雑な例を扱ってみよう.一行の文字列を空白文字の類で分離して望みの型にキャス
トへ変換して返す関数という例である.

\begin{lstlisting}[language={C++},caption=トレイトの複合, label=trait_cmpl]
fn parse_from_line<T: FromStr>(line: &str) -> Result<Vec<T>, Box<Error>>
where T::Err: Error + 'static { ... }
\end{lstlisting}

T: FromStrというのはT型から文字列型へ変換出来る事を示すトレイトであり,
またwhere句での追加の制約は,FromStrトレイトによって定義される「変換に
失敗した場合に出すエラーの型Err」が,Rust全体で汎用的に使われているエラーの
トレイトErrorを実装している事,加えて,T::Errの値が独立した寿命を持てる('static)事を要求してい
る.完全に理解する必要は無いが,オブジェクト指向におけるインターフェースと異なり,
階層構造を持たず,複数の制約を自由に組み合わせられる事に注目して欲しい.

更に,今回は踏み入らないが,トレイトは静的ディスパッチと動的ディスパッチを切り替える事が出来る.
言い換えれば型の解決のコストをコンパイル時,実行時のどちらで払うかが
コードに現れるということであるし,
書く際はそれがすこぶる意識させられる.
C++でもtempleteを用いた静的ディスパッチと仮
想関数を用いた動的ディスパッチ(オブジェクトのポリモーフィズム)があるが,
Rustではトレイトオブジェクトという言語機能によって統合されており,
実装側は共通でトレイトを実装するのみで良く,使用時に動的静的を選択する事が出来る.

ここまでで型クラスでは?と思った一部の諸兄,残念ながらHigherKindType対応では無いためFunctor等は努力が必要である.
RustはLLVMに依る最適化を受ける言語ではあるものの,モナ度の高まりはコンパイル時,実行時共にコストが蓄積しがちな代物であるし,
ゼロコスト抽象化と記述力向上を共に実現する言語機能として
健全なマクロやコンパイラプラグイン\footnote{ユーザが書いたRustの任意の
  コードがプラグインという形でコンパイル時に走るようになる. \\
  構文の拡張や,コンパイル時twitterも可能.但し現在nightlyの機能である.}
があるため,諦めるのが賢明だろう.
これらの機能については入門してからのお楽しみである.


\subsection{所有権システム}
ゼロコスト抽象化を担う象徴的な機能のもう1つが所有権システムである.
ガーベジコレクタの無いRustがコンパイル時の処理のみでメモリ安全性を得るための手法であり,
具体的には幾つかの概念,機能をまとめてこう記述される.

\subsubsection{所有権そしてムーブセマンティクス}
所有権とはあるリソースをどのスコープが所有し利用出来るかという権利であり,
また,所有しているスコープが終了する時,
そのリソースがヒープにあろうとメモリを開放しなければならないという義務でもある.
例えば変数を宣言した際にそのスコープは変数の所有権を得たことに成る.

Rustは全てのリソースについて所有権が1つだけである事を要求する.
ここで生じてくるのがムーブセマンティクスである.ムーブとは,
所有権を移す処理であり,
Rustでは代入や関数へ引数を渡すといった「渡す行為」がデフォルトでムーブとなる.
例えば貴方がprintln!マクロに整数が入った変数を渡したとしよう.
すると変数はムーブされ,参照する事は出来なくなる.(それ以降に参照を試みる
とコンパイル時エラー)
この必要性はヒープにアロケートされた値について考えると分かりやすい.
ヒープに値がアロケートされている場合,
我々はそのポインタをスタックにアロケートして参照に用いる.
ムーブセマンティクスが無い場合,代入や関数適用の際に
値を自由に操作出来る存在が複数居ることになり,
何時メモリを開放すれば良いのか分からなくなってしまうのだ.

これらはC++にて醸成された概念であり,
ガーベジコレクタの無い言語にてメモリ安全性を保つ為の基本的な概念である.

\subsubsection{参照と借用}
しかし現実問題参照が無いプログラミングは非常に冗長なものになってしまう.
そこでリソースの所有権を共有するのではなく,
条件付きで所有権を借用し,参照とミュータブルな参照(以降mut参照)を使えるようにする.
後述するが,Rustはデフォルトでイミュータブルを採用している為値も参照も基本変更不
能である.
以下が借用のルールである.
\begin{enumerate}
  \item 全ての参照は所有者のスコープより長く存続してはならない
  \item 1つのスコープでは,次の内どちらかのみが起き得る.
    \begin{enumerate}
      \item リソースに対する1つ以上の参照が存在する
      \item リソースに対するただ1つのみのmut参照が存在する
    \end{enumerate}
\end{enumerate}
各自マルチスレッドプログラミングにおけるdata raceの定義を確認しておくと,この定
義の意味が分かり易いだろう.所有権システムは並列プログラミングも初めから意識しており,
並列化の際に追加の条件が要請されるような事は無い.
このルールに依ってイテレーション中にデータを消してしまう行為や,ダングリングポインタを防ぐことが出来る.

\subsubsection{ライフタイム}
借用のルールは主に所有主のスコープに着目する物であった.
続いて借用した側とされた側の間に働く仕組み,
具体的には所有権を借用した関数や構造体に対し,
参照の有効なスコープを記述する物がライフタイムである.
背景にはML KitやCycloneといった言語で実装されているRegionを用いた手法があるようだが残念ながら筆者の認識は概略程度であり元論文等は未読である.
より意欲的な読者は調べてみるのが良いだろう.

\section{現代の手続き型プログラミング}
Rustはマルチパラダイムのプログラミング言語だ.従って特定のプログラミングのスタイルを強くサポートする形態をとらない.
例えばRustはオブジェクト指向プログラミング言語\footnote{ここではメッセージング指
  向ではなくBjarne Stroustrup氏がC++に取り込みJava等でも用いられている「ユーザ定
  義型としてのオブジェクト指向」に仮定する}とは言えないだろう.
そもそもオブジェクトは無く,再現ができる機能は分割,分散されている.
\begin{itemize}
\item データ型単位ではなくモジュールがアクセスコントロールを行う
\item トレイトの「継承」
  \begin{itemize}
  \item この「継承」は実装しなければならないトレイトが増えるだけ
  \item OOPと異なり階層構造も出来ないし多態性も無い
  \end{itemize}
\item トレイトオブジェクトによる動的ディスパッチがオブジェクトの多態性に近い
  \begin{itemize}
  \item vtableと同様の仕組みを用いている
  \item 一段階のアップキャストとも言えるか
  \end{itemize}
\item メソッドチェーン記法は有る
  \begin{itemize}
  \item 中置演算子を自由に定義できないため,FPでも全体の流れはこれ頼み
  \end{itemize}
\end{itemize}
また,Rustは関数型プログラミング言語とも言えない.相性の悪いポイントは幾つも見当たる.
\begin{itemize}
\item 高階関数と相性が悪い標準のマクロtry!が有る
\item 標準ライブラリには副作用抑制より速度を取っている実装もある
  \begin{itemize}
  \item ミュータブルな参照による引数を介した書き戻しなど
  \end{itemize}
\item Rustは末尾再帰の最適化を保証しない
  \begin{itemize}
  \item LLVMに任せるのみ
  \end{itemize}
\item 中置演算子は自由に定義できない
  \begin{itemize}
  \item メソッドチェーン記法を使う
  \item 一部演算子はトレイトを実装することでオーバーロード可能
  \end{itemize}
\end{itemize}

無論我々は問題に対して効果的なモデルを構築しプログラミングを行うため,
そういった意味合いで局所的に用いる事は有益だが,常に模倣するのは得策と言えないだろう.

ではRustはどのようなプログラミング言語なのか?
私はRustを現代の手続き型プログラミング言語だと考えている.

\subsection{デフォルトイミュータブル}
今更話す事でも無いと思うが,コンテクスト云々と偉そうな事を放言してしまった為説明
する.
プログラムを書く際,全ての変数がデフォルトで不変であるという性質を指す.つまり,
通常の変数は変更する事が出来ず,特殊な修飾子を付けた変数のみ破壊的代入が可能にな
る.プログラムが複雑になる程,どの部分が書き換えられ得るのかを把握する事がバグを抑えこむ上で
重要となってくるが,デフォルトでミュータブルな変数の言語では不変である事を示す修
飾子が氾濫し可読性が低下するため,Rustではデフォルトイミュータブルが採用されてい
る.それでも速度にシビアな領域に取り組むこの言語では計算量優先でミュータブルな参照を用い
る事が多いが,前述の通り借用のルールによってミュータブルな参照は扱いが厳しいため,
入門直後は矢鱈とミュータブルな参照を用いるべきでは無いだろう.

\subsection{式指向}
多くの手続き型プログラミング言語ではif文が有名だろう.文は式の中に組み込む事が出
来ない.
以下はC言語での条件に応じた初期化である.

\begin{lstlisting}[language={C},caption=Cにおけるif文,label=c_if]
int cond = 5;

int res = 0;

if (cond == 5) {
    res = 10;
} else {
    res = 15;
}
\end{lstlisting}
式指向な言語,Rustでは以下のようになる.
\begin{lstlisting}[language={C++},caption=Rustにおけるif式,label=rust_if]
let cond = 5;
let res = if cond == 5 { 10 } else { 15 };
\end{lstlisting}
この些末なコードに意味はない.文にはスコープの隔たりがあり,余計な副作用の発生や,表現が途中で切断されるといった問題に注目して欲しい.

Rustではletによる変数宣言を除き単体での文は存在しない.基本的に式でプログラミングを構成していく事から公式はRustを「式指向」な言語と称している.
しかし文が使えない訳では無く,セミコロンを式に加えて返り値が無い\footnote{これは不正確である.詳細は入門してry}事を明示し,式を文にすることができる.ドキュメントでは.これを式文と表現している.
そのためお馴染み for のようなループも使用可能である.これ以上は詳細に立ち入って
しまうため,如何に整合性が保たれているかは実際に入門して確かめてみて欲しい.
何れにせよ,式指向は余計な副作用や処理のフローを切断せず,また情報量を柔軟に調整
でき,セミコロンの導入により文的にも扱えるため,従来の手続き的コードのほぼ上位互
換と言って差し支え無いだろう.
なお,この特徴的なセミコロンの使い方はSMLやOCamlからの影響であることが明言されている.\cite{inf}

\subsection{enum型とパターンマッチング}
代数的データ型,直和型,タグ付き共用体,他の言語では色々な呼ばれ方をしているが,
Rustではそれをenum型と呼んでいる.
enum型は宣言した複数の選択肢(Rustではヴァリアントと呼んでいる)の何れかを取る型だ.
但しこの型を開く時,Rustは必ずヴァリアントの網羅性を要求する.
値を取り出そうとする際に全てのバリアントの扱いを網羅できておらず,かつデフォルトのパターンも存在しない場合,コンパイルエラーとなるのである.

enum型によって得られる恩恵として
,まず \textbf{nullの無い世界} が開かれる.
値が無い可能性を有する型はenumに依って簡単に表現できるからだ.
\begin{lstlisting}[language={C++},caption=Option型,label=option_t]
enum Option<T> {
    None,
    Some(T),
}
\end{lstlisting}
この場合Noneか,T型の値を1つ包むSomeの何れかがOption$<$T$>$型である.
利用例として
高階関数を使って条件を設定し,それに合う物を取り出す関数を以下に示そう.
なお,実際にはイテレータという枠組みがあるためこういったコードは書くべきでない.
\begin{lstlisting}[language={C++},caption=条件に応じた値が見つかれば返す関数,label=find_if]
fn find_if<T, F>(v: &Vec<T>, f: F) -> Option<&T>
where F : Fn(&T) -> bool {
    for x in v {
        if f(x) { return Some(x); };
    }
    None
}

fn main() {
    let input = vec![1,2,3];
    match find_if(&input, |x| *x > 2) {
        None => println!("No match found..."),
        Some(x) => println!("Found a match!: {}", x),
    }
}
\end{lstlisting}
補足すると,where句の部分は2引数目のFに関数であるトレイトFnを要求している.
この関数の呼び出し元は11行目であり,
その対応する第2引数には $|x| *x > 2$ という記述が有るが,これがRustのクロージャである.
Rustにおいては関数自体も()演算子をオーバーロードするトレイトであり,
where句を用いて制約する事で静的ディスパッチ出来る.
クロージャがヒープアロケートであり動的ディスパッ
チである言語は多いが,Rustではクロージャがとる環境を構造体としてスタックにアロケートし,
関数呼び出しを静的ディスパッチする事でコストを削減出来る.

注目してもらいたいのは関数のシグネチャにOptionが明記される事と,
main関数内のmatch式である.match式内で$=>$という記号の左辺に有るのが「パターン」
であり,このパターンが網羅されている事をRustは要求する.
match式はこのパターンに合う右辺の式を評価する.
分岐が自然と網羅され,頑強さが担保されていくのは明らかだろう.

エラーの際に情報を値として返したいならば,
\begin{lstlisting}[language={C++},caption=Result型,label=result_t]
enum Result<T, E> {
    Ok(T),
    Err(E),
}
\end{lstlisting}
がOptionと共に標準ライブラリで定義されている.\footnote{私個人としてはエラーに限らず「どちら
  かの値」を表すHaskellのEitherの方が優れた命名だと考えている. \\
  type aliasがあるため,EitherにResultという別名を与えれば意味付け上も問題ないだろう.}

Quick Sort,Hoare論理,CSPの作者として知られるTony Hoare氏はnullを10億ドル相当の過ちと称した.
NullPointerExeptionの悪名たるやここで補足するまでも無いだろうが,
根本的原因は殆どの値に入り込む可能性があり,かつどの関数がnullを返すのか型に現れ
ないことにある.
だがnullを廃しenum型を導入すれば,高々数行の定義によって問題は霧散する.
少なくともRustにおいては,気付かぬ間に這いよるnullの影に怯える必要はまるで無いのだ.

また,昨今の言語ではnull対策の為だけの言語機能も有るが,
enum型の表現力は非常に強く,それだけには留まらない.
再帰的に扱えば線形リストや,
\begin{lstlisting}[language={C++},caption=線形リスト,label=list_t]
enum List<T>{
    Nil,
    Cons(T, Box<List<T>>)
}
\end{lstlisting}
赤黒木等も以下のように簡単に型付けが出来る,後はこの型を操作する関数を書い
ていけば良い.
\begin{lstlisting}[language={C++},caption=赤黒木,label=abt_t]
enum RB{R, B}
enum RBTree<V>{
    Leaf,
    Node(RB, V, Box<RBTree<V>>, Box<RBTree<V>>)
}
\end{lstlisting}

型の直和と直積を使いこなす事で,
多くの構造に対して型を付け,
安全に無駄なく取り扱える事が実感できる
だろう.

なお,此処で頻繁に用いられているBox$<$T$>$はヒープにアロケートした値へのポインタ
型だと捉えて差し支え無い.
Rustの標準ライブラリは実行速度やリソースにタイトな領域にフォーカスしており,
値がスタック上にあるかヒープ上にあるか,
何らかの実行時チェックに依る保証が有るかといった特徴が明確に型に現れるよう設計されている.


\section{Rustの門を叩く}
Rustは執筆時点でバージョン1.10.0であり,ライブラリの一部がunstableであるが,言語自体の破壊的な変更は見られなくなった.
現在は実用的な言語を目指し邁進を続けており,既にゲームエンジン,機械学習ライブラリ等が実装され,
特にRustによるOS開発\cite{redox}には注目が集まった.無論これらはほぼ乃至純Rustによる実装である.

バージョン1.0から1周年であるが非常に多くのドキュメントが書かれ,現在も精力的に更新が続けられている.
従って今回は詳細な入門を書く事は避け,
如何なる要請や知見に基づいて彼らがRustへ至ったのか,その道筋の一端を述べようと試みた.
\footnote{残念ながらRustの並列プログラミングを始めとして幾つかの篇を筆者の能力や期限の都合上掲載する事が出来なかった.
  \\ 但し一つ言っておくとwikipedia jaの並列アクターモデルや軽量タスクといった記述は現在では誤りであり参照してはならない.}

今や門は眼前に有る.貴方がそれを叩き,理想のプログラミングを探求する一助と成れたならば幸いである.

\begin{thebibliography}{9}
\bibitem{most_loved} stackoverflow Developer Survey Results 2016 \\
  \url{https://stackoverflow.com/research/developer-survey-2016}
\bibitem{rustup} rustup: the Rust toolchain installer \\
  \url{https://github.com/rust-lang-nursery/rustup.rs}
\bibitem{rustdoc_ja} rust-lang-ja/the-rust-programming-language-ja \\
  \url{https://github.com/rust-lang-ja/the-rust-programming-language-ja}
\bibitem{nomicon} The Rustnomicon - The Dark Arts of Advanced and Unsafe Rust Programming \\
  \url{https://doc.rust-lang.org/nomicon/}
\bibitem{inf} The Rust Reference - Appendix: Influences \\
  \url{https://doc.rust-lang.org/reference.html#appendix-influences}
\bibitem{redox} Redox - Your Next(Gen) OS \\
  \url{http://www.redox-os.org/}
\end{thebibliography}
